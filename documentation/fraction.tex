\documentclass{article}
\usepackage[utf8]{inputenc} % Включаем поддержку UTF8
\usepackage[russian]{babel} % Включаем пакет для поддержки русского языка
\usepackage{listings} % Включаем пакет для подсветки синтаксиса программного кода
\title{Модуль fraction}
\author{Baboon}
\date{Декабрь 2016}
\begin{document}

\maketitle

\lstset{language=C++}

\section{Назначение}

Шаблонный класс, реализует класс рациональной дроби --- поле рациональных дробей над кольцом, которое является шаблонным параметром.

\section{Примеры}

Класс рационального числа-дроби:

\begin{lstlisting}
plls::fraction<int>
\end{lstlisting}

\section{Методы}

Конструкторы:

\begin{lstlisting}
plls::fraction<int> a;  // a == 0
plls::fraction<int> b(-4);  // b == -4
plls::fraction<int> c(2, 3);  // c == 2/3
plls::fraction<int> d(4, -6);  // c == -2/3
\end{lstlisting}

Числитель и знаменатель дроби:

\begin{lstlisting}
plls::fraction<T>::Numerator();
plls::fraction<T>::Denominator();
\end{lstlisting}

Арифметические операторы:

\begin{lstlisting}
const fraction<T> operator+(const fraction<T>& lhs, const fraction<T>& rhs);
const fraction<T> operator-(const fraction<T>& lhs, const fraction<T>& rhs);
const fraction<T> operator*(const fraction<T>& lhs, const fraction<T>& rhs);
const fraction<T> operator/(const fraction<T>& lhs, const fraction<T>& rhs);
\end{lstlisting}

арифметические операторы работают, также, с типом $T$, допустимо неявное преобразование $T$ к plls::fraction<T>.

\end{document}

